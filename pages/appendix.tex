\section*{A Anwendungsaufnahmen}
\appendixfigure{01-settings}{Übersicht der App-Einstellungen.}
\vspace*{\fill}
\appendixfigure{02-create}{Popover-Formular zum Erstellen eines neuen Projektes.}
\vspace*{\fill}
\appendixfigure{03-project}{Übersicht eines ausgewählten Projektes (erkennbar in der Seitenleiste).}
\vspace*{\fill}
\appendixfigure{04-address}{Formular zum Erstellen einer neuen Adresse, die im Projekt verfügbar wird.}
\appendixfigure{05-province}{Auswahl für das Provinzfeld einer Adresse, nachdem bei der Länderauswahl das Land Deutschland gewählt wurde.}
\appendixfigure{06-map}{Übersicht des Karteneditors mit der Werkzeugleiste auf der linken Seite.}
\appendixfigure{07-ruler}{Verwendung des Ruler-Werkzeuges (erkennbar am dunklen Balken).}
\appendixfigure{08-polygon}{Zeichnen eines Polygons in den Kartendaten.}
\appendixfigure{09-unit}{Bearbeiten eines Unit-Features im Karteneditor.}
\appendixfigure{10-change}{Auswahl zum Wechsel des Feature-Typs.}
\appendixfigure{11-label}{Formular zum Anlegen eines weiteren Eintrags für das Name-Feld des Features.}
\appendixfigure{12-labels}{Übersicht der Einträge für den Namen eines Features.}
\appendixfigure{13-venue}{Formular der Einstellungen eines Venue-Features mit ausgewählter Adresse.}
\clearpage
\section*{B Quellcode}
Zur besseren Nachvollziehbarkeit kann der komplette Quellcode der entwickelten Anwendung unter \url{https://github.com/dennisprudlo/indoor-architect} eingesehen und heruntergeladen werden.
