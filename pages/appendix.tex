%
% Bibliography
\printbibliography
\addtableofcontentsitem{Literaturverzeichnis}
\clearpage

%
% Glossary
\printglossary
\addtableofcontentsitem{Glossar}
\clearpage

%
% Attachments
\chapter*{Anhang}
\addtableofcontentsitem{Anhang}
\section*{A Anwendungsaufnahmen}
\appendixfigure{01-settings}{A.1 Übersicht der App-Einstellungen}
\vspace*{\fill}
\appendixfigure{02-create}{A.2 Popover-Formular zum Erstellen eines neuen Projektes}
\vspace*{\fill}
\appendixfigure{03-project}{A.3 Übersicht eines ausgewählten Projektes (erkennbar in der Seitenleiste)}
\vspace*{\fill}
\appendixfigure{04-address}{A.4 Formular zum Erstellen einer neuen Adresse, die im Projekt verfügbar sein wird}
\appendixfigure{05-province}{A.5 Auswahl für das Provinzfeld einer Adresse}
\appendixfigure{06-map}{A.6 Übersicht des Karteneditors mit der Werkzeugleiste auf der linken Seite}
\appendixfigure{07-ruler}{A.7 Verwendung des Ruler-Werkzeugs (erkennbar am dunklen Balken)}
\appendixfigure{08-polygon}{A.8 Zeichnen eines Polygons in den Kartendaten}
\appendixfigure{09-unit}{A.9 Bearbeiten eines Unit-Features im Karteneditor}
\appendixfigure{10-change}{A.10 Auswahl zum Wechsel des Feature-Typs}
\appendixfigure{11-label}{A.11 Formular zum Anlegen eines weiteren Eintrags für das Name-Feld des Features}
\appendixfigure{12-labels}{A.12 Übersicht der Einträge für den Namen eines Features}
\appendixfigure{13-venue}{A.13 Formular der Einstellungen eines Venue-Features mit ausgewählter Adresse}
\clearpage
\section*{B Quellcode}
\label{sec:appendixb}
Zur besseren Nachvollziehbarkeit kann der komplette Quellcode der entwickelten Anwendung unter \url{https://github.com/dennisprudlo/indoor-architect} eingesehen und heruntergeladen werden.
\section*{C Usability Test Aufgabenbogen}
\label{sec:appendixc}
Erledige alle Aufgaben in der aufgelisteten Reihenfolge und Bewerte im Anschluss die Benutzung der Anwendung.
\begin{description}
	\item[A1] Erstelle ein neues Projekt mit dem Namen \emph{Usability Test Projekt}.
	\item[A2] Aktualisiere die Beschreibung des erstellen Projektes zu \emph{Beispielprojekt für einen Usability Test}.
	\item[A3] Füge eine neue Adresse hinzu und wähle Hamburg als Ort. Die restlichen Eingaben können frei gewählt werden.
	\item[A4] Öffne den Kartenbearbeitungsmodus und zeichne ein Sechseck. Notiere die angezeigte \emph{Feature ID} und wechsle den \emph{Feature Type} zu \emph{Venue}.
	\item[A5] Vergebe einen beliebigen Namen und wähle die in Aufagbe 3 erstellte Adresse für das Objekt.
	\item[A6] Miss die Entfernung zwischen Hamburg und Berlin und notiere das Ergebnis.
\end{description}
\begin{description}
	\item[Bei welchen der sechs Aufgaben war unklar, was getan werden muss?]
	\item[Beschreibe bei welchen Aufgaben du Schwierigkeiten hattest und welche.]
\end{description}

%
% Declaration of Authorship
\chapter*{Eidesstattliche Versicherung}
Ich versichere hiermit, dass ich die vorliegende Bachelorthesis selbstständig und ohne fremde Hilfe
angefertigt und keine andere als die angegebene Literatur benutzt habe. Alle von anderen Autoren
wörtlich übernommenen Stellen wie auch die sich an die Gedankengänge anderer Autoren eng anlehnenden
Ausführungen meiner Arbeit sind besonders gekennzeichnet.\\[1ex]
Diese Arbeit wurde bisher in gleicher oder ähnlicher Form keiner anderen Prüfungsbehörde vorgelegt
und auch nicht veröffentlicht.\\[3ex]
Berlin, den \today\\[20ex]
\line(1,0){200}\\
Dennis Prudlo
\clearpage
