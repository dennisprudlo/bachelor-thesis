\documentclass[a4paper, 12pt, twoside]{report}
\linespread{1.3}

\input preprocessors/packages
\input preprocessors/definitions
\input preprocessors/page

\title{Konzeption und Realisierung einer mobilen Anwendung zur Erstellung von Indoor Maps}
\author{Dennis Prudlo}
\date{\today}

%
% Glossary entries
\makeglossaries
\newglossaryentry{library}{
	name={Programmbibliothek},
	plural={Programmbibliotheken},
	description={Eine Sammlung von Programmcode und Funktionen, die sich in eine Anwendung einbinden lassen. Programmbibliotheken sind üblicherweise keine eigenständigen Programme}
}
\newglossaryentry{sdk}{
	name={Software Development Kit},
	description={Sammlung von Werkzeugen, Compilern, Entwicklungsumgebungen und Programmbibliotheken, die zum Entwickeln von Software benötigt werden}
}
\newglossaryentry{i18n}{
	name={Internationalisierung},
	description={In der Entwicklung der Vorgang die Software so vorzubereiten und aufzubauen, dass ein einfaches Hinzufügen von Übersetzungen in andere Sprachen möglich ist}
}
\newglossaryentry{l10n}{
	name={Lokalisierung},
	description={Das Durchführen einer Übersetzung der Anwendung in eine andere Sprache bzw. Kultur}
}
\newglossaryentry{protocol}{
	name={Protokoll},
	description={In der Swift-Entwicklung – in anderen Sprachen auch als Interface bekannt – eine Zusammenfassung von deklarierten Funktionen, ohne Implementation. Wenn eine Klasse oder eine andere Struktur ein Protokoll implementiert, müssen die deklarierten Funktionen in dieser Klasse implementiert werden}
}

\newcounter{currentpage}

\begin{document}

	%
	% Include the thesis titlepage
	\input pages/cover

	%
	%	FRONT MATTER
	\pagenumbering{roman}
	\pagestyle{plain}

	\input pages/preface
	\input pages/abstract

	%
	% Table of contents
	\tableofcontents
	\clearpage

	%
	% List of figures
	\listoffigures
	\addtableofcontentsitem{Abbildungsverzeichnis}
	\clearpage

	%
	% List of tables
	\listoftables
	\addtableofcontentsitem{Tabellenverzeichnis}
	\clearpage

	%
	% Table of acronyms
	\chapter*{Abkürzungsverzeichnis}
	\addtableofcontentsitem{Abkürzungsverzeichnis}
	\begin{acronym}
		\acro{geojson}[GeoJSON]{Geospatial JavaScript Object Notation}
		\acro{imdf}[IMDF]{Indoor Mapping Data Format}
		\acro{json}[JSON]{JavaScript Object Notation}
		\acro{mvc}[MVC]{Model-View-Controller}
		\acro{sdk}[SDK]{Software Development Kit}
		\acro{uuid}[UUID]{Universally Unique Identifier}
	\end{acronym}
	\clearpage

	\setcounter{currentpage}{\arabic{page}}

	%
	%	MAIN MATTER
	\pagenumbering{arabic}
	\pagestyle{fancy}
	\setcounter{page}{\thecurrentpage}

	\input chapters/01-introduction
	\input chapters/02-theory-and-basics
	\input chapters/03-analysis
	\input chapters/04-conception
	\input chapters/05-implementation
	\input chapters/06-conclusion
	\clearpage
	\setcounter{currentpage}{\arabic{page}}

	%
	%	REAR MATTER
	\pagenumbering{Roman}
	\pagestyle{plain}
	\setcounter{page}{\thecurrentpage}

	%
	% Bibliography
	\printbibliography
	\addtableofcontentsitem{Literaturverzeichnis}
	\clearpage

	%
	% Glossary
	\printglossary
	\addtableofcontentsitem{Glossar}
	\clearpage

	%
	% Attachments
	\chapter*{Anhang}
	\appendix
	\addtableofcontentsitem{Anhang}
	\clearpage

	%
	% Declaration of Authorship
	\chapter*{Eidesstattliche Versicherung}
	Ich versichere hiermit, dass ich die vorliegende Bachelorthesis selbstständig und ohne fremde Hilfe
	angefertigt und keine andere als die angegebene Literatur benutzt habe. Alle von anderen Autoren
	wörtlich übernommenen Stellen wie auch die sich an die Gedankengänge anderer Autoren eng anlehnenden
	Ausführungen meiner Arbeit sind besonders gekennzeichnet.\\[1ex]
	Diese Arbeit wurde bisher in gleicher oder ähnlicher Form keiner anderen Prüfungsbehörde vorgelegt
	und auch nicht veröffentlicht.\\[3ex]
	Berlin, den \today\\[20ex]
	\line(1,0){200}\\
	Dennis Prudlo
	\clearpage

\end{document}
