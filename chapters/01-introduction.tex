\chapter{Einleitung}

\section{Motivation}
% Globalisierte Welt\\
Der Tourismus und die Reiselust der Menschen steigt kontinuierlich im globalen Ausmaß. Waren es im Jahr 1950 international noch insgesamt \SI{25}{\mio} Ankünfte von Touristen, so konnten im Jahr 2015 schon \SI{1.186}{\mrd} Ankünfte verbucht werden \parencite{GLA2017}. Laut der World Tourism Organization wird davon ausgegangen, dass die Ankünfte bis 2030 auf \SI{1.8}{\mrd} ansteigen. Das Zurechtfinden in fremden Orten ist ebenfalls nicht mehr so wie vor 70 Jahren. Heutzutage nutzen \SI{77}{\prc} aller Smartphone Nutzer regelmäßig Navigations-Apps. Es sei viel einfacher eine App zu öffnen und den Weg von Punkt A zu Punkt B zu finden, als eine Person zu fragen \parencite{PAN2018}.%
%
Von den Benutzern einer Navigations-App schauen sich \SI{36}{\prc} die Wegbeschreibung an und informieren sich über die Route, bevor sie sich auf den Weg machen \parencite{PAN2018}. Die Wegbeschreibungen sind sehr detailliert, hören allerdings oft bei den Eingängen von Gebäuden auf. Sogenannte ``Indoor Maps`` – die ein Pendant zu den Kartendaten für den Außenbereich sind – werden noch nicht in allen Ländern unterstützt und falls doch, finden sich nur vereinzelt Gebäude, die Indoor Maps anbieten. Google Maps bietet Indoor Maps in 62 Staaten an \parencite{GOO2020}, Apple Maps zählt einzelne Indoor Maps Standorte auf, wie Flughäfen und Einkaufszentren. Allein in Deutschland wurden erst Anfang 2020 die ersten Einkaufszentren in Apple Maps hinzugefügt, dabei wird Deutschland in der Verfügbarkeitsübersicht für Einkaufszentren immer noch nicht aufgelistet \parencite{OES2020}.%
%
\section{Problemstellung}
% Nicht einfach Indoor Kartendaten bereitzustellen\\
% Gebäude von außen generisch angezeigt ist nicht schwer (Umrisse der Anlage)\\
% Innenleben eines Gebäudes ist bei jedem Gebäude unterschiedlich\\
% Gebäude mit unterschiedlichem Zweck haben unterschiedliche Typen von Räumen und Fluren\\
% Gebäude können in allen Formen vorkommen (blockartig, pyramidenartig, rund, etc.)\\
% Nicht immer ist es ein rechteckiges hochhaus, bei dem die Etagen gerade nach oben verlaufen\\
% Es können Balkone, Überhänge, Brücken über Straßen, etc. teil der Anlage sein\\
% Eine einheitliche Lösung zu finden, um Indoor Maps abzubilden ist nicht einfach\\

\section{Zielsetzung}
Das Ziel dieser Bachelorthesis ist es eine mobile Anwendung zu entwicklen, mit der das Erstellen von Indoor Maps einfach und benutzerfreundlich ist. Dabei sollen die in der vorherigen Sektion genannten Probleme berücksichtigt und die Anforderungen umgesetzt werden. Im Laufe der Thesis werden die Anforderungn an ein System zur Erstellung von Indoor Maps aufgestellt und die aktuellen Lösungen anhand dieser Anforderungen analysiert. Anschließend werden Herausforderungen, Probleme und mögliche Verbesserungen der aktuellen Lösungen erkannt und dokumentiert. Diese Verbesserungsmöglichkeiten werden darauffolgend in die entwickelte App implementiert, um so ein möglichst intuitives und benutzerfreundliches Erlebnis zu schaffen.%
%
\section{Signifikanz von Kartendaten}
% Kartendaten Decken alles ab\\
% Parks, Points of Interests etc.\\
% Ansichten für Satellit und Map\\
% ÖPNV von Bus/Tram bis zu internationalen Zügen\\
%
\section{Verwendung von Indoor Maps}
% Einkaufszentren, Stadien, Flughäfen, Bahnhöfen
