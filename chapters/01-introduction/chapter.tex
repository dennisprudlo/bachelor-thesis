\chapter{
	\de{Einleitung}
	\en{Introduction}
}

\section{
	\de{Motivation}
	\en{Motivation}
}
% Globalisierte Welt\\
Der Tourismus und die Reiselust der Menschen steigt kontinuierlich im globalen Ausmaß. Waren es im Jahr 1950 international noch insgesamt \SI{25}{\mio} Ankünfte von Touristen, so konnten im Jahr 2015 schon \SI{1.186}{\mrd} Ankünfte verbucht werden \parencite{GLA2017}. Laut der World Tourism Organization wird davon ausgegangen, dass die Ankünfte bis 2030 auf \SI{1.8}{\mrd} ansteigen. Das Zurechtfinden in fremden Orten ist ebenfalls nicht mehr so wie vor 70 Jahren. Heutzutage nutzen \SI{77}{\prc} aller Smartphone Nutzer regelmäßig Navigations-Apps. Es sei viel einfacher eine App zu öffnen und den Weg von Punkt A zu Punkt B zu finden, als eine Person zu fragen \parencite{PAN2018}.%
%
Von den Benutzern einer Navigations-App schauen sich \SI{36}{\prc} die Wegbeschreibung an und informieren sich über die Route, bevor sie sich auf den Weg machen \parencite{PAN2018}. Die Wegbeschreibungen sind sehr detailliert, hören allerdings oft bei den Eingängen von Gebäuden auf. Sogenannte "Indoor Maps" – die ein Pendant zu den Kartendaten für den Außenbereich sind – werden noch nicht in allen Ländern unterstützt und falls doch, finden sich nur vereinzelt Gebäude, die Indoor Maps anbieten. Google Maps bietet Indoor Maps in 62 Staaten an \parencite{GOO2020}, Apple Maps zählt einzelne Indoor Maps Standorte auf, wie Flughäfen und Einkaufszentren. Allein in Deutschland wurden erst Anfang 2020 die ersten Einkaufszentren in Apple Maps hinzugefügt, dabei wird Deutschland in der Verfügbarkeitsübersicht für Einkaufszentren immer noch nicht aufgelistet \parencite{OES2020}.%
%
\section{
	\de{Problemstellung}
	\en{Problem definition}
}
% Abkürzungen mit \ac{imdf}
% Nicht einfach Indoor Kartendaten bereitzustellen\\
% Gebäude von außen generisch angezeigt ist nicht schwer (Umrisse der Anlage)\\
% Innenleben eines Gebäudes ist bei jedem Gebäude unterschiedlich\\
% Gebäude mit unterschiedlichem Zweck haben unterschiedliche Typen von Räumen und Fluren\\
% Gebäude können in allen Formen vorkommen (blockartig, pyramidenartig, rund, etc.)\\
% Nicht immer ist es ein rechteckiges hochhaus, bei dem die Etagen gerade nach oben verlaufen\\
% Es können Balkone, Überhänge, Brücken über Straßen, etc. teil der Anlage sein\\
% Eine einheitliche Lösung zu finden, um Indoor Maps abzubilden ist nicht einfach\\

\section{
	\de{Zielsetzung}
	\en{Target setting}
}
% Analyse der aktuellen Lösungen zum Erstellen von Indoor Maps\\
% Probleme und Herausforderungen der Lösungen herauslesen\\
% Eine mobile Anwendung erstellen, die das generieren von Indoor Maps mäglichst einfach und bedienerfreundlich macht\\

\section{
	\de{Signifikanz von Kartendaten}
	\en{Significance of mapping data}
}
% Kartendaten Decken alles ab\\
% Parks, Points of Interests etc.\\
% Ansichten für Satellit und Map\\
% ÖPNV von Bus/Tram bis zu internationalen Zügen\\

\section{
	\de{Verwendung von Indoor Maps}
	\en{Use of Indoor Maps}
}
% Einkaufszentren, Stadien, Flughäfen, Bahnhöfen
