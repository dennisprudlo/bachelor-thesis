\usepackage[T1]{fontenc}
\usepackage[ngerman]{babel}
\usepackage{fontspec}

%
% Used to include graphics
\usepackage{graphicx}

%
% Used to build a title
\usepackage{titling}

%
% Used to set spacing
\usepackage{setspace}

%
% Used to style the headers and footers
\usepackage{fancyhdr}

%
% Used to format latex generated titles
\usepackage{titlesec}

%
% Used for predefining colors to use
\usepackage{color}

%
% Used to create a table of acronyms
\usepackage{acronym}

%
% Used to format captions
\usepackage{caption}
\captionsetup{font=footnotesize}

%
% Used to define the layout of the pages
\ifdefined\isrelease
	\usepackage[
		a4paper,
		width=150mm,
		top=30mm,
		bottom=30mm,
		bindingoffset=0mm
	]{geometry}
\else
	\usepackage[
		showframe, % used to show the frame while debugging
		a4paper,
		width=150mm,
		top=30mm,
		bottom=30mm,
		bindingoffset=0mm
	]{geometry}
\fi

%
% Used for syntax highlighted codes
\usepackage[outputdir=build]{minted}

%
% Used for bibliography rendering
\usepackage{csquotes}
\usepackage[
	backend=biber,
	style=apa,
	citestyle=authoryear,
	citetracker=false,
	uniquelist=false,
	sorting=none,
	sortcites=true,
	block=none,
	indexing=false,
	citereset=none,
	isbn=true,
	url=true,
	doi=true,
	natbib=true
]{biblatex}
\addbibresource{references.bib}		% Add the bibliography resource

%
% Used for SI-Unit rendering
\usepackage{siunitx}
\sisetup{locale = DE}

%
% Used to obtain TeX logo commands
\usepackage{metalogo}

%
% Used for linking the table of contents items and references
\ifdefined\isrelease
	\usepackage[
		hidelinks,
		pdfusetitle,
		pdfsubject={Konzeption und Realisierung einer mobilen Anwendung zur Erstellung von Indoor Maps},
		pdflang={de},
		pdfkeywords={Indoor Mapping, Indoor Maps, iOS, Application, Mobile Application, iPad, JSON, GeoJSON, GIS},
		pdfproducer={XeLaTeX \the\eTeXversion\eTeXrevision-\the\XeTeXversion\XeTeXrevision},
		pdfcreator={XeLaTeX}
	]{hyperref}
\else
	\usepackage{hyperref}
	\hypersetup{
		citebordercolor={0 0 1}
	}
\fi

%
% Used to generate glossary entries
\usepackage[automake]{glossaries}

%
% Used to display a directory tree
\usepackage{dirtree}
\DTsetlength{0.3em}{1em}{0.3em}{0.4pt}{1.6pt}
