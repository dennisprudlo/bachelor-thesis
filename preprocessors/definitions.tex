%
% Colors
\definecolor{gray75}{gray}{0.75}
\definecolor{gray25}{gray}{0.25}

\definecolor{redL}{RGB}{255,59,48}
\definecolor{redDL}{RGB}{255,69,58}
\definecolor{redH}{RGB}{215,0,21}
\definecolor{redDH}{RGB}{255,105,97}

%
% Fonts
\newfontfamily\sfmono{SF Mono}[NFSSFamily=SFMono, Scale=1]
\setmonofont[Scale=0.9]{SF Mono}

%
% Use monospaced line numbers in listings only when the release flag is set
\ifdefined\isrelease
	%\renewcommand{\theFancyVerbLine}{\sfmono\textcolor{gray25}{\scriptsize\oldstylenums{\arabic{FancyVerbLine}}}}
\fi

%
% Inserts a chapter title into the table of contents
\newcommand{\addtableofcontentsitem}[1]{
	\addcontentsline{toc}{chapter}{#1}
}

%
% Set the \paragraph indentation length to 0
\setlength{\parindent}{0pt}

%
% Create paragraph break command
\newcommand{\pbreak}{\\[1em]}

\usemintedstyle{lovelace}
\setminted{fontsize=\small,baselinestretch=1,fontfamily=SFMono}

\setlength\bibitemsep{1.3ex}		% Vertical item spacing
\DeclareFieldFormat{apacase}{#1}	% Prevent mis-format for sentences beginning with an umlaut.

\DeclareSIUnit{\prc}{\text{\%}}
\DeclareSIUnit{\mio}{\text{Mio.}}
\DeclareSIUnit{\mrd}{\text{Mrd.}}
\DeclareSIUnit{\eur}{\text{€}}

%
% Insert a code listing
\newcommand{\codelisting}[4]{
	\begin{listing}[ht]
		\inputminted[fontsize=\scriptsize,obeytabs=true,tabsize=4,linenos,xleftmargin=1.7em]{#4}{listings/#3}
		\caption{#1}
		\label{#2}
	\end{listing}
}

\newcommand{\appendixfigure}[2]{
\begin{figure}[h]
	\centering
	\includegraphics[scale=0.25]{images/appendix/#1}
	\caption*{#2}
	\label{fig:#1}
\end{figure}
}

\newcommand{\analysisresults}[5]{
\subsubsection{Preisliche Stemmbarkeit}
#1
\subsubsection{Schnelle Einarbeitungszeit}
#2
\subsubsection{Leichte und nachvollziehbare Bedienung}
#3
\subsubsection{Mobile Nutzung}
#4
\subsubsection{Fazit}
#5
}

\let\oldsection\section
\makeatletter
\newcounter{@secnumdepth}
\RenewDocumentCommand{\section}{s o m}{%
  \IfBooleanTF{#1}
    {\setcounter{@secnumdepth}{\value{secnumdepth}}% Store secnumdepth
     \setcounter{secnumdepth}{0}% Print only up to \chapter numbers
     \oldsection{#3}% \section*
     \setcounter{secnumdepth}{\value{@secnumdepth}}}% Restore secnumdepth
    {\IfValueTF{#2}% \section
       {\oldsection[#2]{#3}}% \section[.]{..}
       {\oldsection{#3}}}% \section{..}
}
\makeatother

%\makeatletter \def\@pnumwidth{40pt}\makeatother
